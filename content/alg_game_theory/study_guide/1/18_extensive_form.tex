\subsection{Hry v rozšířeném tvaru}

\begin{definition}\label{def:extensive_form}
Hra v rozšířeném tvaru je orientovaný strom, kde vrcholy reprezentují stavy hry. 
Hra začíná v kořenu stromu a končí v listech, kde každý hráč získá výplatu. 
Každý vrchol jenž není listem je rozhodovacím uzlem a hrany mezi vrcholy odpovídají hrané akci v daném stavu hry, ze kterého hrana vede, do stavu hry, který máme, když hráč zahraje danou akci. 

Hratelné akce jsou tedy takové, které vycházejí z rozhodovacího uzlu.
\end{definition}

\begin{definition}\label{def:perfect_info}
    Hrají-li hráči hru v rozšířeném tvaru \ref{def:extensive_form} a vždy znají rozhodovací uzel, ve kterém zrovna jsou (tedy znají i historii jak jsme se k němu dobrali), tak pak máme \textit{hru s dokonalou informací}. 
\end{definition}
\begin{definition}\label{def:imperfect_info}
    Rozdělíme-li rozhodovací uzly na množiny informací, kde množina informací vždy patří jednomu hráči a mají stejné hratelné akce, a každý hráč zná jen v jaké množině informací je, a ne nutně jak se tam dostal a kde konkrétně, pak se jedná o \textit{hru s nedokonalou informací}. 

    Pro hráče $i$ definujeme $H_i$ jako množinu informačních množin daného hráče, a pro množinu informací $h \in H_i$ máme $C_h$ jako množinu akcí v $h$.
\end{definition}
\begin{definition}\label{def:behavioral_strategy}
Behaviorální strategie hráče $i$ je ditribuce pravděpodobností na $C_h$ pro každé $h \in H_i$. 
Tedy je to strategie, kde výběr akce u dané informace je nezávislá na ostatních. 
\end{definition}
\begin{definition}\label{def:sequence}
    Posloupnost tahů hráče $i$ až do uzlu $t$ je $\sigma_i(t)$ je jednoznačná cesta z kořene do $t$. 
    Prázdnou značíme $\emptyset$.
\end{definition}
\begin{definition}\label{def:perfect_recall}
Hra s dokonalou pamětí, je taková, že žádný z hráčů nezapomene žádnou informaci o tazích které zatím zahrál. 

Hráč $i$ má dokonalou paměť, právě tehdy když pro každé $h \in H_i$ a libovolné uzly $t,t' \in h$ máme $\sigma_i(t) = \sigma_i(t')$, pak $\sigma_h$ značí jednoznačnou posloupnost vedoucí ke každé $t \in h$. 

Mají-li dokonalou paměť všichni hráči, pak i hra je s dokonalou pamětí. 
\end{definition}

\begin{theorem}[Khunova věta]\label{thm:khun}
    Ve hře s dokonalou pamětí \ref{def:perfect_recall}, každá smíšená strategie \ref{def:mixed_strategy} daného hráče a ekvivalentní behaviorální strategie \ref{def:behavioral_strategy} jsou vzájemně nahraditelné. 
\end{theorem}
