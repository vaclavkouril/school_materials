\subsection{Polytopy nejlepších odpovědí}
\begin{definition} 
\label{def:best_response_polyhedrom}
Polyedr nejlepších odpovědí hráče $1$ ve hře $G$ je polyedr 
$$
\bar{P} = \{(x,v)\in \R^m \times \R: x \geq 0, 1^Tx = 1, N^Tx \leq 1v\}
$$ 
a pro hráče $2$ máme obdobný:
$$
\bar{Q} = \{(y,u)\in \R^n \times \R: y \geq 0, 1^Ty = 1, Mx \leq 1u\}
$$
\end{definition}
Tedy pro body $(x,v)$ a $(y,u)$ máme výplatní funkci pro hráče 1., kde pro libovolnou smíšenou strategii máme 
$$ 
u_1(s) = x^T M y \leq x^T1u = u \sum_{i \in A_1} x_1 = u. 
$$ 
\begin{definition}[Značka]
\label{def:label}
Mějme bod $(x,v) \in \bar{P}$, pak takový bod má \textit{značku} $i \in A_1 \cup A_2$, když buď $i \in A_1, x_i = 0$, nebo když $i \in A_2$ a $(N^T)_i x = v$ (tedy $i$-tá řádka matice $N^T$). 

Obdobně pro bod $(y,u) \in \bar{Q}$, který má \textit{značku} $i \in A_1 \cup A_2$, když buď $i \in A_2, x_i = 0$, nebo když $i \in A_1$ a $(M)_i y = u$. 
\end{definition}
\begin{theorem}
\label{thm:complete_label_nash}
Profil strategie $s$ je Nashovým ekvilibriem hry $G$, právě tehdy když $((x,v), (y,u))\in \bar{P} \times \bar{Q}$ je plně značený, jinými slovy, každá značka \ref{def:label} $i \in A_1 \cup A_2$ se ukáže v množině značek buď $(x,v)$ nebo $(y,u)$.
\end{theorem}
\begin{proof}[Důkaz věty \ref{thm:complete_label_nash}]
    Dle věty o podmínce nejlepší odpovědi \ref{thm:best_response}, máme že pro každého hráče $i \in P$, se smíšenou strategií \ref{def:mixed_strategy} $s_i$ je nejlepší odpovědí na $s_{-i}$, právě tehdy když všechny čisté strategie \ref{def:pure_strategy} v doméně smíšené strategie $s_i$ \ref{def:support_strategy} jsou nejlepšími odpověďmi na $s_{-i}$. 
    Chybějící značka $i$ by přesně znamenala, že čistá strategie u hráče $1$ ($x_i >0$) by nebyla nejlepší odpovědí ($(M)_i y < u$), to samé pro hráče $2$. 

    Pokud máme každou značku u nějakých $s_1, s_2$, tak jsou to vzájemně nejlepší odpovědi, protože každá čistá strategie buď je nejlepší odpovědí nebo není v příslušné doméně strategie.
\end{proof}

\begin{definition}[Normalizovaný polytop nejlepších odpovědí]
\label{def:norm_best_response_poly}
Předpokládejme, že $M$ a $N^T$ jsou nezáporné a nemají nulový sloupec. 
Normalizovaný polytop nejlepších odpovědí pro hráče $1$ ve hře $G$ je 
$$
P = \{x\in \R^m: x \geq 0, N^Tx \leq 1\}
$$ 
a pro hráče $2$ máme
$$
Q = \{y\in \R^n: y \geq 0, Mx \leq 1\}.
$$
\end{definition}

\begin{theorem}
\label{thm:corollary_normalized_complete_label}
Nechť $s = (s_1, s_2)$ je profilem strategie ve hře $G$ a buď $x$ a  $y$ vektory smíšených strategií \ref{def:mixed_strategy} $s_1$ a $s_2$.
Pak $s$ je Nashovým ekvilibriem hry $G$, právě tehdy když bod $(x/u_2(s), y/u_1(s)) \in P \times Q \setminus \{(0,0)\}$ je kompletně označen \ref{def:label}.
\end{theorem}
\begin{proof}[Komentář k \ref{thm:corollary_normalized_complete_label}]
    Zavedeme-li si transformaci $\bar{P}$ na $P \setminus \{0\}$, pomocí $(x,v)\rightarrow x/v$ tak pak je toto jen přeformulovaná \ref{thm:complete_label_nash}. 
\end{proof}

\begin{algorithm}
    \floatname{algorithm}{Algoritmus}
    \algrenewcommand\algorithmicrequire{\textbf{Vstup: }}
    \algrenewcommand\algorithmicensure{\textbf{Výstup: }}
    \caption{Výčet bodů}
    \label{alg:vertex_enum}
    \begin{algorithmic}
        \Require  nedegenerovaná bimaticová hra $G$ \ref{def:nondegen_bimatrix}
        \Ensure  všechna Nashova ekvilibria $G$
        
        \For{každý pár (x,y) bodů z $(P \setminus \{0\}) \times (Q \setminus \{0\})$}
            \State když (x,y) je plně označen tak ho vrátíme jako $((x/(1^Tx),(y/(1^Ty))$ Nashovo ekvilibrium
        \EndFor
    \end{algorithmic}
\end{algorithm}

\begin{proof}[Vysvětlení algoritmu \ref{alg:vertex_enum}]
    Pomocí předešlého faktu \ref{thm:corollary_normalized_complete_label} máme jednoduchý algoritmus, kde je na konci jen nutná normalizace abychom dostaly vektory smíšené strategie.
    Problém je množství bodů a jejich kombinace takže je to zase exponenciální algoritmus v $O(dNv)$, kde máme-li $m = n$, tak vrcholů je $c^n$ s konstantou $c > 1$, kde $c$ je omezeno z geometrie a celkem bodů bude maximálně $2.9^n$. 
\end{proof}
