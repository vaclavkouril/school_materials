\subsection{Třída PPAD}
\begin{definition}
\label{def:FNP}
Třída FNP je třída problémů, kde problém dostane jako vstup instanci problému z NP a výstupem je buď NE, nebo existuje-li řešení, tak vrátí řešení instance tohoto problému. 

Funkcionální SAT je FNP-complete problém. 
\end{definition}
\begin{definition}\label{def:NASH}
Nechť NASH je problémem hledání Nashova ekvilibria \ref{def:nash_equilibrium} v bimaticových hrách. 
\end{definition}

\begin{theorem}
\label{thm:np_conp}
Je-li NASH \ref{def:NASH} problém FNP-complete \ref{def:FNP} tak NP = coNP
\end{theorem}
\begin{proof}[Důkaz \ref{thm:np_conp}]
    Existuje-li redukce v polynomiálním čase mezi funkcionálním SAT a NASH, tak máme algoritmus $A$ zobrazující každou SAT formuli $\varphi$ na bimaticovou hru $A(\varphi)$ a algoritmus $B$ převádějící každé Nashovo ekvilibrium \ref{def:nash_equilibrium} $s = (s_1, s_2)$ hry $A(\varphi)$ na validní řešení $B(s_1,s_2)$ SATu $\varphi$, a odpoví NE pro vše ostatní.


    Stačí nám ukázat, že SAT patří do coNP, protože pak NP $\subseteq$ coNP a máme $X \in NP \iff \bar{X} \in coNP$ z definice coNP. 
    Tedy máme-li NP $\subseteq$ coNP, pak $X \in coNP \iff \bar{X} \in NP \Rightarrow \bar{X} \in coNP \iff X \in NP$ a tedy máme i $coNP \subseteq NP$ a tedy $NP = coNP$. 

    Pokud v polynomiálním čase umíme ověřit nesplnitelnost $\varphi$ pomocí obrazu $A(\varphi)$, pak SAT je v coNP. 
    Díky Nashově větě \ref{thm:nash_theorem} víme, že $A(\varphi)$ má Nashovo ekvilibrium $s = (s_1, s_2)$, pro důkaz nesplnitelnosti $\varphi$ jsou nutné dvě podmínky.
    Nejdřive zda řešení $A(\varphi)$ je Nashova rovnováha, umíme ověřit díky polynomiálnosti algoritmu $A$ a podmínce nejlepší odpovědi \ref{thm:best_response}.
    Dále překontrolujeme, že $B(s_1, s_2)$ vrací NE. 
    Splňuje-li $(s_1,s_2)$ obě podmínky, tak jsme dokázali nesplnitelnost SAT a tedy jsme v polynomiálním čase dokázali SAT $\in$ coNP
\end{proof}

\begin{definition}
\label{def:eof}
END-OF-LINE problém je následující: Pro orientovaný graf $G$, kde každý bod má maximálně jednoho předka a jednoho následníka. 
My ale takový graf nemáme na vstupu (třeba se nevejde do paměti atd.), ale máme jen bod a funkci, která v polynomiálním čase spočte předka, či následníka (existují-li). 

O tomto problému se dá také přemýšlet tak, že máme bod, který je řešení, a máme funkci $f$, která umí řešení vylepšit, tedy určuje následníka jako další řešení a my chceme to nejlepší řešení. 
\end{definition}
\begin{definition}
\label{def:PPAD}
Třída PPAD ("Polynomial Parity Argumets on Directed graphs") je třída problémů redukovatelných na END-OF-LINE problém \ref{def:eof}. 
\end{definition}

\begin{theorem}
\label{thm:nash_ppad}
NASH \ref{def:NASH} je PPAD-complete \ref{def:PPAD}
\end{theorem}
