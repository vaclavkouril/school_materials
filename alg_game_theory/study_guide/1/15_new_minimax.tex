\subsection{Nový důkaz Minimax věty}
\begin{theorem}
\label{thm:minimax_new}
Mějme zero-sum ($C_1(a) - C_2(a) = 0$ pro všechna $a \in A$) hru $G = \{\{1,2\}, A, C\}$, kde hodnota hry je $v$, ostatní definice z \ref{thm:minimax}. 
Když hráč $i\in \{1,2\}$ hraje $G$ pomocí online algoritmu $ON_i$ s vnější lítostí \ref{def:external_regret} $R$, pak jeho kumulovaná ztráta je $L_{ON_i}^T$ splňuje 
\[
    L_{ON_i}^T \leq Tv + R
\]
\end{theorem}

\begin{proof}[Důkaz věty \ref{thm:minimax_new}]
    Symetrií to stačí ukázat pro hráče $1$. 
    Nechť $p^t_{2,j}$ je pravděpodobnost toho, že druhý hráč vybere $j$-tou akci z $A_2$ v kroce $t$. 
    Zadefinijeme si vypozorovaný vektor smíšené strategie \ref{def:mixed_strategy} dle frekvence daných akcí hraných hráčem $2$ jako $q = (q_1,q_2,\dots,q_{\abs{A_2}})$, kde $q_j = \sum^T_{t=1} p^t_{2,j}/T$. Víme že pro libovolný vektor smíšené strategie $q$ druhého hráče má hráč $1$ akci $a_i$, že $\E[b_2\sim q]{C(a_i, b_2)} \leq v$, tedy mpoužijeme-li je pokaždé, tak máme $L^T_{ON_i,\min} \leq L^T_{ON_i,i} \leq vT$. 
    A vzhledem k předpokladu o vnější lítosti $R$ algoritmu $ON_i$, tak máme $L^T_{ON_i} \leq L^T_{ON_i,\min} + R \leq  vT +R$
\end{proof}

\begin{proof}[Nový důkaz minimax věty \ref{thm:minimax}]
    Rozdělíme důkaz na dvě nerovnosti a začneme s $\max_x \min_y x^T My \leq \min_y \max_x x^T M y$, protože je jednodušší. 
    Protože $x^*$ je optimální strategií prvního hráče hraje-li první a může si vybrat $x^*$ i když hraje druhý. 
    Je jen horší jít první ale stále platí nerovnost. 

    Všechny $m_{i,j} \in [-1,1]$, jinak je přeškálujeme. 
    Vybereme $\epsilon \in (0,1]$ a spustíme bezlístostnou dynamiku \ref{alg:no_regrets} $T$ dostkrát aby oba hráči měli očekávanou vnější lítost (regret) $\epsilon$. 
    Využijeme-li třeba algoritmu polynomiálních vah \ref{alg:poly_weights} tak stačí $T \geq 4\ln \max\{m,n\}/\epsilon^2$, kde $m = \abs{A_1}, n = \abs{A_2}$, díky tomu že ztráta při kroku je $2\sqrt{\ln \max\{m,n\}/T}$. 
    
    Nechť $p_1,p_2,\dots,p_T$, $q_1,q_2,\dots,q_T$ jsou smíšené strategie hráče 1 a 2, která jsou doporučena algoritmem. 
    Máme výplatní vektory $Mq^t$ a $-(p^t)^T M$ pro oba hráče. 
    Mějme $\bar{x} = \frac{1}{T} \sum^T_{t=1} p^t$ průměrnou smíšenou strategií hráče 1 a pro hráče 2 $\bar{y} = \frac{1}{T} \sum_{t=1}^T p^t$. 
    Průměrná očekávaná výplata je $\frac{1}{T} \sum_{t=1}^T (p^t)^T M q^t$. 

    Pro čistou strategii \ref{def:pure_strategy} $a_i$ zavedeme smíšenou strategii $e_i$, která má všude $0$, kromě $i$-tého indexu, kde je jedna. 
    Protože vnější lítost je maximálně $\epsilon$ u hráče jedna tak máme 
    \[
        e_i^T M \bar{y} = \frac{1}{T} \sum^T_{t=1} e_i^T M q^t \leq \frac{1}{T} \sum_{t=1}^T (p^t)^T M q^t + \epsilon = v +\epsilon 
    \]
    pro všechna $i\in \{1,\dots,m\}$. 
    Vzhledem k tomu, že každá smíšená strategie je lineární kombinací $e_i$ tak linearitou střední hodnoty máme $\forall x \in S_i: \sum_{t=1}^T x^T M \bar{y} \leq v+\epsilon$. 
    Pro hráče dva máme symetrický argument tak máme $\forall y \in S_2: \sum_{t=1}^T \bar{x}^T M y \geq v-\epsilon$. 

    Dáme-li vše dohromady tak máme 
    \[
        \max_{x\in S_1} \min_{y\in S_2} x^T M y \geq \min_{y\in S_2} (\bar{x})^T M y \geq v-\epsilon 
    \]
    \[
        \geq max_{x\in S_1} x^T M \bar{y} -2\epsilon \geq  \min_{y\in S_2} \max_{x\in S_1} x^T M y -2\epsilon.
    \]
    Pro $T\rightarrow \infty$ máme $\epsilon \rightarrow 0$ a tedy nerovnost platí. 
\end{proof}

