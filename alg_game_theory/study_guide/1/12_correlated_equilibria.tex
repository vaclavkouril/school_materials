\subsection{Korelováná ekvilibria}
\begin{definition}[Korelované ekvilibrium]
\label{def:correlated_equiv}
Pro hru v normální formě \ref{def:normal_form_game} $G=(P,A,u)$ mějme pravděpodobnostní distribuci $p$ na $A$. 
Tedy $p(a) \geq 0$ pro každé $a \in A$ a $\sum_{a\in A} p(a) = 1$. 
Distribude $p$ je \textit{korelované ekvilibrium} v $G$, když 
$$ 
\sum_{a_{-i} \in A_{-i}} u_i(a_i'; a_{-i}) p(a_i; a_{a_{-i}}) \leq \sum_{a_{-i} \in A_{-i}} u_i(a_i; a_{-i}) p(a_i; a_{a_{-i}})  
$$
pro všechny hráče $i$ a všechny čisté strategie \ref{def:pure_strategy} $a_i, a_i' \in A_i$.
\end{definition}
Korelované ekvilibrium je náhodné přiřazení doporučení akcí hráčů takové, že se nikomu nevyplatí nejednat podle doporučení. 

\begin{theorem}
\label{thm:nash_correl}
    V každé hře $G$ v normální formě \ref{def:normal_form_game} máme pro každé Nashovo ekvilibrium \ref{def:nash_equilibrium} odpovídající korelované ekvilibrium \ref{def:correlated_equiv}. 
\end{theorem}
\begin{proof}[Důkaz věty \ref{thm:nash_correl}]
    Nechť $s* = (s*_1, s*_2, \dots, s*_n)$ je profil strategií, mějme funkci $p_{s^*} = \prod^n_{j=1} s_j^* (a_j)$ pro všechny strategie $a = (a_1, a_2, \dots, a_n) \in A$. 
    Jasně je $p_{s^*} \geq 0$ a indukcí dle $n$ se dá i ukázat, že $\sum_{a\in A} p_{s^*}(a) =1$ a je to tedy pravděpodobnostní distribuce. 

    Teď už je zbývá ukázat, že pro Nashovo ekvilibrium $s^*$ je $p_{s^*}$ korelované ekvilibrium. 
    Zafixujeme si $i \in P$ a $a_i, a_i' \in A_i$. Předpokládejme $s^*(a_i) > 0$ a tedy že čistá strategie $a_i$ je v doméně strategie $s^*$ \ref{def:support_strategy}. 
    Z definice $u_i$ a $p_{s^*}$ máme 
    $$ 
    u_i(a_i'; s^*_{-i}) = \sum_{a_{-i} \in A_{-i}} u_i(a_i'; a_{-i}) \prod_{j=1, j\neq i}^n s^*_j(a_j) = \frac{1}{s^*_i(a_i)} \sum_{a_{-i} \in A_{-i}} u_i(a_i'; a_{-i}) p_{s^*} (a_i; a_{-i})
    $$
    analogicky máme to samé pro $a_i$
    $$ 
    u_i(a_i; s^*_{-i}) = \sum_{a_{-i} \in A_{-i}} u_i(a_i; a_{-i}) \prod_{j=1, j\neq i}^n s^*_j(a_j) = \frac{1}{s^*_i(a_i)} \sum_{a_{-i} \in A_{-i}} u_i(a_i; a_{-i}) p_{s^*} (a_i; a_{-i})
    $$
    Dle podmínky nejlepší odpovědi \ref{thm:best_response} a toho, že $s^*$ je Nashovo ekvilibrium, tak máme že $a_i$ v doméně $s^*$ je také nejlepší odpovědí na $s_{-i}$ a tedy $u_i(a_i'; s^*_{-i}) \leq u_i(a_i; s^*_{-i})$ a tedy 
    $$ 
    \frac{1}{s^*_i(a_i)} \sum_{a_{-i} \in A_{-i}} u_i(a_i'; a_{-i}) p_{s^*} (a_i; a_{-i}) \leq \frac{1}{s^*_i(a_i)} \sum_{a_{-i} \in A_{-i}} u_i(a_i; a_{-i}) p_{s^*} (a_i; a_{-i})
    $$ 
    a stačí nám obě strany přenásobit $s^*_i(a_i)$ má hledanou rovnici. 
    Navíc když $s^*_i(a_i) = 0$ tak $p_{s^*}(a_i; a_{-i}) = 0$ pro všechna $a_{-i} \in A_{-i}$ a tedy máme korelované ekvilibrium.
\end{proof}

\begin{theorem}
\label{thm:convex_correlated}
V každé hře v normální formě \ref{def:normal_form_game} $G= (P,A,u)$, je konvexní kombinace korelovaných ekvilibrií \ref{def:correlated_equiv} opět korelované ekvilibrium. 
\end{theorem}

\begin{proof}[Důkaz \ref{thm:convex_correlated}]
   Mějme $p,p'$ korelovaná ekvilibria a ukážeme, že $p'' = tp + (1-t) p'$ je také korelované ekvilibrium. 
   $$ 
   \sum_{a_{-i} \in A_{-i}} u_i(a_i; a_{-i})p''(a_i; a_{-i}) = \sum_{a_{-i} \in A_{-i}} u_i(a_i; a_{-i})( tp(a_i; a_{-i}) + (1-t)p'(a_i; a_{-i})) \geq 
   $$

   $$ 
   \sum_{a_{-i} \in A_{-i}} u_i(a_i'; a_{-i})( tp(a_i; a_{-i}) + (1-t)p'(a_i; a_{-i})) = \sum_{a_{-i} \in A_{-i}} u_i(a_i'; a_{-i})p''(a_i; a_{-i}) 
   $$
   pro všechny hráče a čisté strategie \ref{def:pure_strategy} $a_i, a_i' \in A_i$
\end{proof}
