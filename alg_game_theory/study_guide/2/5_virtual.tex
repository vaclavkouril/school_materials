\subsection{Maximalizace virtuálního sociálního užitku}
\begin{theorem}\label{thm:max_virtual}
  Nechť $F$ je regulární distribuce pravděpodobností s hustotou $f$, a vituálním užitkem $\varphi$, $F_1, \dots,F_n$ jsou vzájemně nezávislé distribuce pravděpodobností $n$ hráčů takové, že se rovnají. 
  Pak Vickreyho aukce \ref{def:vickrey} s rezervou $\varphi^{-1} (0)$ maximalizuje zisk \ref{def:revenue}. 
\end{theorem}
\begin{proof}[Důkaz věty \ref{thm:max_virtual}]
    Dle věty \ref{thm:virtual} je maximalizace zisku stejná jako maximalizace virtuálního sociálního užitku.
    Abychom ho maximalizovali, tak vybíráme $x(b)$ pro každý vstup $b$, ale ne pravděpodobnostní distribuce. 
    Mějme virtuální sociální užitek maximalizující mechanismus, který má omezení $\sum^n_{i=1}x_i(b) \leq 1$ pro každé $b$, tedy chceme dát výhru hráči $i$ s nejvyšší $\varphi(b_i)$. 
    Virtuální užitek ale může býti negativní, a pak je nejlepší nikomu nic neprodávat. 

    Máme tedy alokační pravidlo maximalizující virtuální užitek. 
    Teď stačí ukázat, že je monotónní, protože pak aplikujeme Myersonovo lemma \ref{thm:myerson} a máme DSIC $(x,p)$ mechanismus. 
    Díky tomu, že $F$ je regulární, tak $\varphi$ sdílená všemi je čistě rostoucí a náš mechanismus je ekvivalentní Vickreyho aukci s rezervou $\varphi^{-1}(0)$. 
\end{proof}
