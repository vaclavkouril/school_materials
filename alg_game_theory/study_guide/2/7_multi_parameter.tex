\subsection{Několika parametrové prostředí}
\begin{definition}
    \label{def:multi_param}
    Několika parametrové navrhování mechanismů je, kde každý $i$ má různé ohodnocení pro různé předměty. 
    Tedy 
    \begin{itemize}
        \item n účastníků
        \item konečná množina $\Omega$ výsledků
        \item pro každého hráče $i$ ohodnocení $v_i(\omega) \geq 0$ pro $\forall \omega \in \Omega$
    \end{itemize}
\end{definition}

\begin{theorem}[Vickrey-Clarke-Groves věta]
    \label{thm:vickrey_clarke_groves}
    V každém několika parametrovém prostředí návrhu mechanismů \ref{def:multi_param} je DSIC \ref{def:awesome} mechanismus na maximalizaci socialního užitku. 
\end{theorem} 
\begin{proof}[Důkaz věty \ref{thm:vickrey_clarke_groves}]
   Předpokládáme, že se každý odhalí a vybere se výsledek $\Omega$. 
   Chceme maximalizovat sociální užitek, tedy jsme svázáni vybráním alokačního pravidla, které ho maximalizuje. 
   Dány $b = ((b_1(\omega))_{\omega \in \Omega},\dots,(b_n(\omega))_{\omega \in \Omega})$ máme 
   \[
       x(b) = argmax_{\omega in \Omega}\sum^n_{i=1} b_i(\omega). 
   \]

   Definujeme pravidlo platby, tak aby nikomu nebylo jedno, jak na tom jsou ostatní tedy 
   \[
   p_i(b) = \max_{\omega \in \Omega} \left\{ \sum^n_{j=1, j\neq i} b_j(\omega) \right\} - \sum^n_{j=1, j\neq i} b_j(\omega^*)
   \]
   pro každého $i$, kde $\omega^* = x(b)$, je výsledkem našeho alokačního pravidla $x$ pro daná $b$. 
   První člen je vlastně sociální užitek ostatních, když vynecháme $i$. 
   Druhý člen je s $i$-tým účastníkem. 
   Z definice tedy $(x,p)$ maximalizuje sociální užitek. 
    
   Zbývá už jen to, zda je to DSIC, snažíme se tedy ukázat, že každý maximalizuje svůj užitek $v_i(x(b)) - p_i(b)$ nastavením $b_i(\omega) = v_i(\omega)$ pro všechny $\omega$. 
   Dá se ukázat nezápornost $p_i(b)$ a je $\leq b_i(\omega^*)$. 
   Tedy pravdomluvní mají nezáporný užitek. 

   Zafixujeme si $i$ a ostatní $b_{-i}$. 
   Když $x(b)= \omega^*$, pak užitek je 
   \[
    v_i(\omega^*) - p_i(b) = \left( v_i(\omega^*) + \sum^n_{j=1, j\neq i} b_j(\omega^*) \right) - \max_{\omega \in \Omega} \left\{ \sum^n_{j=1, j\neq i} b_j(\omega) \right\} 
   \]
   a vzhledem k tomu, že druhý člen je na $b_i$ nezávislý, tak je nutné maximalizovat ten první, navíc nemá ani, jak ovlivnit $\omega^*$, protože přímo mechanismus $(x,y)$ vybírá $\omega^*$. 
   VCG mechanismus vybere $\omega^*$ podle  $x(b) = argmax_{\omega in \Omega}\sum^n_{i=1} b_i(\omega)$, tak aby součet nabídek byl maximalizován. 
   $i$ je na tom tak že chce vybrat 
   \[
   argmax_{\omega in \Omega} \left\{ v_i(\omega) + \sum^n_{j=1, j\neq i} b_j(\omega) \right\}
   \]
   když nabídky jsou pravdivé tak máme $argmax_{\omega in \Omega}\sum^n_{i=1} b_i(\omega)$. 
   Tedy máme $(x,p)$ DSIC, protože jiná strategie než býti pravdomluvným se nevyplácí tolik. 
\end{proof}
