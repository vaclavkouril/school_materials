\subsection{Nashova věta}
\begin{theorem}[Nashova věta]
\label{thm:nash_theorem}
Každá hra v normální formě \ref{def:normal_form_game} má Nashovo ekvilibrium \ref{def:nash_equilibrium}.
\end{theorem}
Pro větu si půjčujeme poměrně dost z oblasti matematické analýzy a celý důkaz je jen topologický a ne konstruktivní, tedy dokázáním věty stále nemáme tušení, jak taková ekvilibria vypadají, ani jak, či zda vůbec se jich dopočítáme, jen víme, že nějaká jsou. 
Takový fakt nám pak posouvá problém hledání (nejen) Nashových ekvilibrií mimo třídy $P$, či $NP$, které jsou z definice rozhodovací a odpověď na otázku zda existuje Nashovo ekvilibrium je velmi nudná a díky Nashově větě je vždy ANO. 

Pro důkaz je důležitá vlastnost \textit{kompaktnosti}, tedy bavíme-li se o $X \subseteq \R^d$ je $X$ \textit{kompaktní} je-li uzavřená a omezená, tedy máme nějaké konstanty které ji omezují a uzavřená je pokud limity jejich posloupností jsou vždy uvnitř takové množiny. 
\textit{Konvexita} podmnožiny $Y \subseteq \R^d$ je $Y$, kde $\forall x,y \in Y$ platí $tx + (1-t)y \in Y, t \in [0,1]$. 
Také se hodí definice \textit{simplexu}, tedy pro $n$ affině nezávislých bodů $x_1, x_2, \dots, x_n$ je $(n-1)$-simplex množina $\Delta_n = \Delta_n (x_1, x_2, \dots, x_n)$ množiny bodů $\{x_1, x_2, \dots, x_n\}$ je konvexní kombinace těchto bodů, tedy
\[
    \Delta_n = \left\{ y | y = \sum_{i = 1}^n t_ix_i, \forall i\in {1,2,\dots,n}, t_i \in [0,1], \sum^n_{i=0} t_i = 1\right\}
\]

\begin{theorem}[Brouwerova věta o pevném bodu]
\label{thm:brouwer_theorem}
Pro $d \in \N$, $K$ je \textit{neprázdná kompaktní konvexní} množina v $\R^d$. Nechť $f: K \rightarrow K$ je spojité zobrazení. Pak $\exists x_0 \in K: f(x_0) = x_0$, tedy $f$ má pevný bod $x_0$. 
\end{theorem}

\begin{theorem}[Lemma o kompaktnosti kartézského součinu kompaktních množin]
\label{thm:compact_lemma}
Pro $n \in \N$ a $d_1,d_2,\dots,d_n \in \N$, nechť $\forall i \in \{1,2,\dots,n\} K_i \in \R^{d_i}$ jsou kompaktní. 
Pak $K_1 \times K_2 \times \cdots \times K_n$ je kompaktní v $\R^{d_1 + d_2 + \dots + d_n}$. 
\end{theorem}
\begin{proof}[Náznak důkazu lemma \ref{thm:compact_lemma}]
Stačí to ukázat pro případ kdy $n =2$, zbytek se induktivně naskládá. 
Omezenost je jednoduchá a tedy katézským součinem nemáme jak uniknou nějaké konstantě, která omezovala $K_1$, či $K_2$, nazveme je $C_1, C_2$ a ty pak zase po složkách omezují nové prvky. 

Uzavřenost je zase v tom, že nemáme jak po dvojicích vytvořit zase nějaké divoké posloupnosti, protože jestli po složkách limity patřily do původních množin, tak ty další složky nás nemohou nijak ovlivnit. 
\end{proof}
Hlavní idea důkazu je, že si pro každého hráče definujeme jeho množinu smíšených strategií jako simplex v reálném prostoru, pak na základě lemmatu \ref{thm:compact_lemma} je množina smíšených strategií kompaktní a konvexní. 
Dodefinujeme si zobrazení $f$, které hráče přesměruje z jeho aktuální strategie na nějakou strategii $s'$, která zohledňuje odpovědi na strategie ostatních. 
Ukáže se, že $f$ je spojitá a z Brouwerovy věty o pevném bodu \ref{thm:brouwer_theorem} najdeme pevný bod který má tedy charakteristiku, že jestli se zobrazuje na sebe tak nemáme motivaci nic zlepšit a tedy jsme narazili vlastně na ekvilibrium $s^*$. 
\begin{proof}[Důkaz Nashovy věty \ref{thm:nash_theorem}]
Mějme hru v normálním tvaru $\ref{def:normal_form_game}$ $G = (P, A, u)$. 
Pro každého hráče $i$ zavedeme simplex $S_i$ na množině jeho čistých strategií $A_i$. 
\[
    S_i = \left\{ s_i \in \R^{\abs{A_i}}: \forall a \in A_i s_i(a) \geq 0, \sum_{a \in A_i} s_i(a) = 1 \right\}
\],
kde $s_i(a)$, značí pravděpodobnost výběru akce $a$ při smíšené strategii $s_i$. 

Dle lemma \ref{thm:compact_lemma} je i $S = S_1 \times S_2 \times \cdots \times S_n$ kompaktní a konvexita je jednoduchá, protože pro $\forall s =(s_1, s_2, \dots, s_n), s' =(s_1', s_2', \dots, s_n') \in S$ a $t \in [0,1]$ máme jistě $ts + (1-t)s' \in S$, kde to se jen po složkách nasčítá a z konvexity původních $S_i$, tedy jednotlivých složek máme konvexitu $S$. 
Zaveďme si vhodné spojité zobrazení $f: S \rightarrow S$. 
Pro to mějme aktuální profil strategií $s =(s_1, s_2, \dots, s_n)$, a pro každého hráče $i$ a každou akci $a_i \in A_i$ definujeme funkci pro měření zlepšení užitku hráče $i$ a tedy 
\[
    \varphi_{i,a_i}(s_i) = \max\{0,u_i(a_i; s_{-i}) - u_i(s) \}.
\]
Funkce $\varphi_{i,a_i}$ jsou spojité, protože jsou složené ze spojité konstantní $0$ a $u_i(a_i;s_{-i}) - u_i(s)$, která je taky spojitá, protože spojitost plyne z linearity $u_i(s) = \sum_{a \in A_i} s(a) u_i(a)$

S použitím $\varphi_{i,a_i}$ můžeme definovat nový profil strategií $s'$, jen je potřeba myslet na normalizování pravděpodobností, aby to stále byl profil strategií $s_i'(a_i) \in [0,1]$ a $\sum_{a_i \in A_i} s_i'(a_i) = 1$
\[
    s_i'(a_i) = \frac{s_i(a_i) + \varphi_{i,a_i}(s_i)}{\sum_{b_i \in A_i} s_i(b_i) + \varphi_{i,b_i}(s)} = \frac{s_i(a_i) + \varphi_{i,a_i}(s_i)}{1 + \sum_{b_i \in A_i} \varphi_{i,b_i}(s)}
\]
pro všechna $i \in P, a_i \in A_i$. Nakonec nastavíme $f(s) = s'$. 

Taková $f$ je spojitá, protože jen sčítáme spojité funkce. Tedy díky Brouwerově větě pevném bodu \ref{thm:brouwer_theorem} získáváme, že $f$ má nějaký pevný bod. 

Teď zbývá ukázat, že $s^*$ je pevným bodem $f$, právě tehdy když $s^*$ je Nashovým ekvilibriem. 

Pokud je $s^*$ pevným bodem, pak pro každého hráče $i \in P$ a akci $a_i \in A_i$ platí $\varphi_{i,a_i}(s^*) = 0$. Tedy žadný z hráčů nemá jak si zvýšit svůj užitek přechodem na jinou strategii. 
Proto $s_i^{*}$ je nejlepší odpovědí na $s_{-i}^{*}$ pro všechna $i$, tedy je to z definice nutně Nashovo ekvilibrium \ref{def:nash_equilibrium}.
\end{proof}
