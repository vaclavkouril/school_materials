\subsection{Princip odhalení}

\begin{definition}\label{def:revel}
  Přímé odhalení je speciální dominantní strategie, že každý pravdivě prozradí vše soukromé mechanismu. 
\end{definition}

\begin{theorem}[Princip odhalení]
    \label{thm:revelation}
    Pro každý několika parametrový mechanismus \ref{def:multi_param} $M$, kde má každý dominantní strategii, tak nezávisle na soukromém ohodnocení máme ekvivalentní mechanismus $M'$, kde každý má dominantní strategii, a to přímé odhalení \ref{def:revel}. 
\end{theorem}
\begin{proof}[Důkaz principu odhalení]
    Pro každého $i$ a jeho $(v_i(\omega))_{\omega \in \Omega}$, nechť $s_i((v_i(\omega))_{\omega \in \Omega})$ je dominantní strategie $i$ v $M$. 

    Zkonstruujeme $M'$, přijmeme uzavřené nabídky $b_1(\omega)_{\omega \in \Omega}, \dots, b_n(\omega)_{\omega \in \Omega}$. 
    
    Pak $M'$ předá $s_1(b_1(\omega)_{\omega \in \Omega}), \dots, s_n(b_n(\omega)_{\omega \in \Omega})$ mechanismu $M$, tedy výsledky jsou pak stejné. 

    Dle přímého odhalení, pokud $i$ má $v_i(\omega)$, pak nabídnutí něčeho jiného než $v_i(\omega)$ může znamenat jen hraní jiné strategie než $s_i(v_i(\omega)_{\omega \in \Omega})$, což ale znamená jen snížení užitku. 
\end{proof}
