\subsection{Bulow-Klempererova věta}

\begin{theorem}[Bulow-Klempererova věta]
\label{thm:bulow_klemperer}
Nechť $F = F_1 = \cdots = F_n$ je regulární distribuce pravděpodobností a nechť $n \in \N$. 
Pak platí 
\[
    \E[v_1,\dots,v_{n+1}\sim F]{Rev(VA_{n+1})} \geq \E[v_1,\dots,v_{n+1}\sim F]{Rev(OPT_{F,n})} 
\]
kde $Rev(VA_{n+1})$ je zisk Vickreyho aukce $VA_{n+1}$ s n+1 účastníky bez rezervy a $Rev(OPT_{F,n})$ je zisk optimální aukce pro $F$ s n účastníky. 
\end{theorem}
\begin{proof}[Důkaz Bulow-Klempererovy věty \ref{thm:bulow_klemperer}]
    Definujme si pomocnou aukci $\mathcal{A}$ s $n+1$ účastníky, tak že simuluje optimální aukci na n účastnících, kde pokud nebyl předmět dán v předešlém kroku, tak se dá n+1-mu zdarma. 
    Máme 
    \[
        \E[v_1,\dots,v_{n+1}\sim F]{Rev(\mathcal{A})} = \E[v_1,\dots,v_{n+1}\sim F]{Rev(OPT_{F,n})}
    \]
    taková aukce vše alokuje

    Je-li $F=\cdots=F_n$ regulární, tak Vickreyho aukce maximalizuje očekávaný zisk přes všechny aukce, které vždy alokují předmět. 
    Z věty o vztahu mezi ziskem a virtuálním sociálním užitkem \ref{thm:virtual} tak máme, že optimální aukce, která alokuje předměty, předá předmět hráči s nejvyšším virtuálním ohodonocením (i když potencielně negativním). 
    Vzhledem k regularitě $F$ tak $\varphi$ stoupá a tedy ten, který má nejvyšší vituální ohodnocení, má i nejvyšší $v_i$. 
    Takže Vickreyho aukce $VA_{n+1}$ má střední hodnotu zisku alespoň tak vysokou jako aukce, která alokuje všechny předměty, tedy 
    \[
        \E[v_1,\dots,v_{n+1}\sim F]{Rev(VA_{n+1})} \geq \E[v_1,\dots,v_{n+1}\sim F]{Rev(\mathcal{A})} = \E[v_1,\dots,v_{n+1}\sim F]{Rev(OPT_{F,n})}
    \]
\end{proof}

